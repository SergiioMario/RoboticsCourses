\documentclass[letterpaper,12pt]{article}
\usepackage[spanish]{babel}
\spanishdecimal{.}
\usepackage[utf8]{inputenc}
\usepackage{graphicx}
\usepackage[top=2.5cm, bottom=2.5cm, left=2.5cm, right=2.5cm]{geometry}

\title{Práctica 1 \\ Motores de DC y el circuito puente H}
\author{M.I. Marco Negrete}
\date{Robots Móviles y Agentes Inteligentes}
\begin{document}
\renewcommand{\tablename}{Tabla}
\maketitle
\section*{Objetivos}
\begin{itemize}
\item Familiarizar al alumno con el uso del sistema operativo Linux, distribución Ubuntu versión 14.04.
\item Aprender los conceptos básicos de ROS (Robot Operating System).
\end{itemize}

\section*{Introducción}
ROS es un \textit{middleware} de código abierto (\textit{open source}) que provee la funcionalidad comúnmente necesaria en el desarrollo de software para robots móviles autónomos, como paso de mensajes y manejo de paquetes. La robot Justina y los demás robots móviles que se usarán durante el resto del curso utilizan ROS como plataforma de desarrollo, por lo que es importante conocerlo.

ROS puede describirse en dos niveles conceptuales: el sistema de archivos y el grafo de procesos.

\textbf{El sistema de archivos.} Se refiere al modo en que están organizados los recursos en disco:
\begin{itemize}
\item \textbf{\textit{Workspace:}} Se refiere a las carpetas que contienen paquetes de ROS.
\item \textbf{Paquete:} Es la principal unidad de organización de software en ROS. Pueden contener nodos, bibliotecas, datasets, archivos de configuración y otros.
\item \textbf{Manifiesto:} Definido por el archivo \texttt{package.xml} en cada paquete. Provee metadatos acerca de cada paquete. 
\item \textbf{Mensaje:} Archivos con extensión \texttt{.msg}. Definen estructuras de datos para el paso de mensajes en ROS.
\item \textbf{Servicio:} Archivos con extensión \texttt{.srv}. Definen estructuras de tipo \textit{request-response}. Utilizan mensajes para dicha definición.
\end{itemize}

\textbf{Grafo de procesos.} Es una red \textit{peer-to-peer} de procesos. Los componentes básicos son:
\begin{itemize}
\item \textbf{Roscore:} Inicializa el sistema ROS: un máster + rosout + un servidor de parámetros.
\item \textbf{Nodos:} Es simplemente un ejecutable que usa ROS para comunicarse con otros nodos.
\item \textbf{Tópicos:} Algo similar a una variable cuyo contenido puede ser compartido entre todos los nodos mediante un patrón de publicación y suscripción. 
\item \textbf{Servicios:} Otra forma de comunicar nodos pero con un patrón de petición y respuesta.
\item \textbf{Servidor de parámetros:} Es un diccionario compartido. Todos los nodos pueden leer y escribir parámetros en tiempo de ejecución.
\end{itemize}

\section*{Desarrollo}

\end{document}