\documentclass[letterpaper,12pt]{article}
\usepackage[spanish]{babel}
\spanishdecimal{.}
\usepackage[utf8]{inputenc}
\usepackage{graphicx}
\usepackage[top=2.5cm, bottom=2.5cm, left=2.5cm, right=2.5cm]{geometry}

\title{Práctica 1 \\ Motores de DC y el circuito puente H}
\author{M.I. Marco Negrete}
\date{Robots Móviles y Agentes Inteligentes}
\begin{document}
\renewcommand{\tablename}{Tabla}
\maketitle
\section*{Objetivos}
\begin{itemize}
\item Familiarizar al alumno con el uso del sistema operativo Linux, distribución Ubuntu versión 14.04.
\item Aprender los conceptos básicos de ROS (Robot Operating System).
\end{itemize}

\section*{Introducción}
ROS es un \textit{middleware} de código abierto (\textit{open source}) que provee la funcionalidad comúnmente necesaria en el desarrollo de software para robots móviles autónomos, como paso de mensajes y manejo de paquetes. ROS puede describirse en dos niveles conceptuales: el sistema de archivos y el grafo de procesos.

\textbf{El sistema de archivos.} Se refiere al modo en que están organizados los recursos en disco:
\begin{itemize}
\item \textbf{\textit{Workspace:}} Se refiere a las carpetas que contienen paquetes de ROS.
\item \textbf{Paquete:} Es la principal unidad de organización de software en ROS. Pueden contener nodos, bibliotecas, datasets, archivos de configuración y otros.
\item \textbf{Manifiesto:} Definido por el archivo \texttt{package.xml} en cada paquete. Provee metadatos acerca de cada paquete. 
\item \textbf{Mensaje:} Archivos con extensión \texttt{.msg}. Definen estructuras de datos para el paso de mensajes en ROS.
\item \textbf{Servicio:} Archivos con extensión \texttt{.srv}. Definen estructuras de tipo \textit{request-response}. Utilizan mensajes para dicha definición.
\end{itemize}

\textbf{Grafo de procesos.} Es una red \textit{peer-to-peer} de procesos. Los componentes básicos son:
\begin{itemize}
\item \textbf{Mater (roscore):}
\item \textbf{Servidor de parámetros:}
\item \textbf{Nodos:}
\item \textbf{Tópicos:}
\item \textbf{Servicios:}
\end{itemize}

\section*{Desarrollo}

\end{document}