\documentclass[letterpaper,12pt]{article}
\usepackage[spanish]{babel}
\spanishdecimal{.}
\selectlanguage{spanish}
\usepackage[spanish,onelanguage,ruled]{algorithm2e}
\usepackage[utf8]{inputenc}
\usepackage{graphicx}
\usepackage{caption}
\usepackage{subcaption}
\usepackage[top=2cm, bottom=2cm, left=2cm, right=2cm]{geometry}
\usepackage{hyperref}
\usepackage{verbatim}
\usepackage{amssymb}
\usepackage{mathtools}
\newcommand\ddfrac[2]{\frac{\displaystyle #1}{\displaystyle #2}}

\title{Práctica 7  \\ Evasión de obstáculos mediante campos potenciales}
\author{M.I. Marco Negrete}
\date{Robots Móviles y Agentes Inteligentes}
\begin{document}
\renewcommand{\tablename}{Tabla}
\maketitle
\section*{Objetivos}
\begin{itemize}
\item Implementar un comportamiento reactivo para evasión de obstáculos mediante campos potenciales.
\item Determinar los parámetros de diseño para obtener un comportamiento satisfactorio.
\item Probar la evasión de obstáculos tanto en simulación como experimentalmete.
\end{itemize}

\section{Marco Teórico}
\subsection{Campos potenciales artificiales}

\subsection{Descenso del gradiente}

\subsection{El sensor láser Hokuyo-URG}

\section{Tareas}

\subsection{Prerrequisitos}
Antes de continuar, actualice el repositorio y recompile:
\begin{verbatim}
   cd ~/RoboticsCourses
   git pull origin master
   cd catkin_ws
   catkin_make
\end{verbatim}


\subsection{Nodo que implementa los campos potenciales}

\section{Evaluación}
\begin{itemize}
\item El código debe estar ordenado.
\item \textbf{Importante: } Si el alumno no conoce su código, NO se contará la práctica.
\end{itemize}

\end{document}