\documentclass[letterpaper,12pt]{article}
\usepackage[spanish]{babel}
\spanishdecimal{.}
\usepackage[utf8]{inputenc}
\usepackage{graphicx}
\usepackage{caption}
\usepackage{subcaption}
\usepackage[top=2.5cm, bottom=2.5cm, left=2.5cm, right=2.5cm]{geometry}
\usepackage{hyperref}
\usepackage{verbatim}

\title{Práctica 5  \\ Cálculo de rutas utilizando A* y celdas de ocupación}
\author{M.I. Marco Negrete}
\date{Robots Móviles y Agentes Inteligentes}
\begin{document}
\renewcommand{\tablename}{Tabla}
\maketitle
\section*{Objetivos}
\begin{itemize}
\item A partir del mapa creado en la práctica 4, calcular una ruta mediante el algoritmo A*.
\item Suavizar la ruta utilizando descenso del gradiente. 
\item Publicar la ruta en un tópico y desplegarla en el visualizador \texttt{rviz}.
\end{itemize}

\section{Introducción}

\section{Desarrollo}
\textbf{Nota:} Para esta práctica se asume que el alumno ejecutó correctamente la práctica 4.

\subsection{Algoritmo de A*}
%Poner aquí el pseudogódigo
\subsection{Suavizado}
%Poner pseudocódigo de suavizado
\subsection{Visualización}
%Instrucciones para agregarla al rviz y ejemplo con imagen de lo que se espera obtener.
\section{Evaluación}
\begin{itemize}
\item La práctica se considera entregada si el mapa representa al ambiente de manera satisfactoria. 
\end{itemize}

\end{document}