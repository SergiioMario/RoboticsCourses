\documentclass[letterpaper,12pt]{article}
\usepackage[spanish]{babel}
\spanishdecimal{.}
\selectlanguage{spanish}
\usepackage[spanish,onelanguage,ruled]{algorithm2e}
\usepackage[utf8]{inputenc}
\usepackage{graphicx}
\usepackage{caption}
\usepackage{subcaption}
\usepackage[top=2cm, bottom=2cm, left=2cm, right=2cm]{geometry}
\usepackage{hyperref}
\usepackage{verbatim}
\usepackage{amssymb}
\usepackage{mathtools}
\usepackage{listings}
\usepackage{color}
\definecolor{backcolour}{rgb}{0.95,0.95,0.92}
\newcommand\ddfrac[2]{\frac{\displaystyle #1}{\displaystyle #2}}
\lstset{backgroundcolor=\color{backcolour}, basicstyle=\footnotesize}
\lstset{xleftmargin=1cm, xrightmargin=1cm, breaklines=true}

\title{Práctica 8 \\ Instalación y configuración de la computadora RaspberryPi}
\author{Laboratorio de Bio-Robótica}
\date{Construcción de Robots Móviles}
\begin{document}
\renewcommand{\tablename}{Tabla}
\maketitle
\section*{Objetivos}
\begin{itemize}
\item Familiarizar al alumno con el uso del software de control de versiones \texttt{git}.
\item Aprender a utilizar el software desarrollado en el Laboratorio de Bio-Robótica para la operación de robots móviles autónomos. 
\end{itemize}

\section{Introducción}

\end{document}

%%% Local Variables:
%%% mode: latex
%%% TeX-master: t
%%% End:
