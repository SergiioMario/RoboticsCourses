\documentclass[letterpaper,12pt]{article}
\usepackage[spanish]{babel}
\spanishdecimal{.}
\selectlanguage{spanish}
\usepackage[spanish,onelanguage,ruled]{algorithm2e}
\usepackage[utf8]{inputenc}
\usepackage{graphicx}
\usepackage{caption}
\usepackage{subcaption}
\usepackage[top=2cm, bottom=2cm, left=2cm, right=2cm]{geometry}
\usepackage{hyperref}
\usepackage{verbatim}
\usepackage{amssymb}
\usepackage{mathtools}
\usepackage{listings}
\usepackage{color}
\definecolor{backcolour}{rgb}{0.95,0.95,0.92}
\newcommand\ddfrac[2]{\frac{\displaystyle #1}{\displaystyle #2}}
\lstset{backgroundcolor=\color{backcolour}, basicstyle=\footnotesize}
\lstset{xleftmargin=1cm, xrightmargin=1cm, breaklines=true}

\title{Práctica 3 \\ Implementación de un nodo de ROS en la tarjeta arduino}
\author{Laboratorio de Bio-Robótica}
\date{Construcción de Robots Móviles}
\begin{document}
\renewcommand{\tablename}{Tabla}
\maketitle
\section*{Objetivos}
\begin{itemize}
\item Familiarizar al alumno con la tarjeta de desarrollo Arduino y sus herramientas de desarrollo.
\item Implementar un nodo de ROS en la tarjeta Arduino usando la biblioteca \texttt{rosserial}.
\item Practicar el uso de publicadores y suscriptores implementados en la tarjeta arduino. 
\end{itemize}

\section{Introducción}

Arduino es una tarjeta electrónica de desarrollo \textit{open-source} basada en la idea del software y hardware fáciles de usar. El hardware consiste en una tarjeta, comúnmente basada en microcontroladores Atmel AVR, con puertos analógicos y digitales de entrada/salida, además de varios módulos de comunicación y control comúnmnte usados en el desarrollo de robots móviles como generador de señales PWM, puerto RS232, comunicación I2C, entre otros. 

El software Arduino consiste en un entorno de desarrollo integrado (IDE), que permite escribir programas y cargarlos en la tarjeta, y un lenguaje de programación (muy parecido al lenguaje C), específico para las tarjetas Arduino. 

La biblioteca \texttt{rosserial} implementa un protocolo para empaquetar mensajes serializados estándares de ROS, además, permite multiplexar múltiples tópicos y servicios sobre dispositivos como puertos seriales o \textit{sockets}. El paquete \texttt{rosserial\_arduino} contiene extensiones específicas de Arduino para correr un cliente de la biblioteca \texttt{rosserial} en una tarjeta Arduino.

Mediante \texttt{rosserial\_arduino} se puede implementar un nodo de ROS en la tarjeta Arduino Uno (que es la que se usará en este curso) que publique o se suscriba a tópicos y que atienda o llame a servicios, sin embargo, existen ciertas limitaciones. Dado que la tarjeta Arduino Uno posee sólo 2 kB de memoria RAM, no es posible enviar mensajes muy largos y el número de publicadores, suscriptores, clientes y servicios es muy limitado. Más adelante se dan instrucciones para no superar la memoria del Arduino Uno. 

\section{Desarrollo}
\subsection{Instalación del IDE de Arduino}

Descargue el IDE de Arduino de la página \url{https://www.arduino.cc/en/Main/Software}. Seleccione la descarga de acuerdo con el sistema que tenga instalado, 32 o 64 bits. 

Descomprima el archivo (se puede hacer mediante un click derecho y la opción \textit{Extract here}), abra una terminal y cambie el directorio de trabajo a la carpeta que se acaba de extraer. Ejecute el archivo \texttt{install.sh}:

Suponiendo que el archivo descargado sea \texttt{arduino-1.8.1-linux64.tar.xz} y la carpeta de descargas esté en \texttt{~/Downloads}, los comandos para instalar el IDE serían:
\begin{lstlisting}[language=bash]
$  cd ~/Downloads      
$  tar xf arduino-1.8.1-linux64.tar.xz     #Descomprime el archivo
$  cd arduino-1.8.1
$  ./install.sh
\end{lstlisting}

Durante la instalación, seleccione siempre las opciones por default. 

\subsection{Instalación de la biblioteca \texttt{rosserial_arduino}}


\subsection{Código de ejemplo}

\subsection{Implementación de un publicador y un subscritor}

\textbf{Nota.} Se asume que el alumno ya tiene instalado Ubuntu 14.04 y ROS Indigo. 

\end{document}

%%% Local Variables:
%%% mode: latex
%%% TeX-master: t
%%% End:
