\documentclass[letterpaper,12pt]{article}
\usepackage[spanish]{babel}
\spanishdecimal{.}
\selectlanguage{spanish}
\usepackage[spanish,onelanguage,ruled]{algorithm2e}
\usepackage[utf8]{inputenc}
\usepackage{graphicx}
\usepackage{caption}
\usepackage{subcaption}
\usepackage[top=2cm, bottom=2cm, left=2cm, right=2cm]{geometry}
\usepackage{hyperref}
\usepackage{verbatim}
\usepackage{amssymb}
\usepackage{mathtools}
\usepackage{listings}
\usepackage{color}
\definecolor{backcolour}{rgb}{0.95,0.95,0.92}
\newcommand\ddfrac[2]{\frac{\displaystyle #1}{\displaystyle #2}}
\lstset{backgroundcolor=\color{backcolour}, basicstyle=\footnotesize}
\lstset{xleftmargin=1cm, xrightmargin=1cm, breaklines=true}

\title{Práctica 6 \\ Lectura de un acelerómetro con la tarjeta arduino e implementación de un filtro pasa-bandas}
\author{Laboratorio de Bio-Robótica}
\date{Construcción de Robots Móviles}
\begin{document}
\renewcommand{\tablename}{Tabla}
\maketitle
\section*{Objetivos}
\begin{itemize}
\item Utilizar el circuito MMA8452Q para medir la aceleración del robot móvil. 
\item Comunicar la tarjeta Arduino Uno con el acelerómetro mediante I2C.
\item Implementar un nodo de ROS en la tarjeta Arduino Uno que publique los valores del acelerómetro. 
\item Utilizar una interfaz gráfica de usuario (GUI) para desplegar los valores del acelerómetro. 
\item Implementar un filtro pasa-bandas para acondicionar la señal del acelerómetro.
\end{itemize}

\section{Introducción}
Dependiendo del nivel de complejidad de la tarea que se pretenda resolver, un robot puede o no necesitar conocer su posición con respecto a algún sistema de referencia. Si la tarea no requiere de un alto nivel cognitivo, el comportamiento inteligente se puede lograr mediante la implementación de varios comportamientos y en general no es necesario conocer la posición del robot. Por el contrario, si la tarea implica la pleneación de rutas o el seguimiento de trayectorias, entonces sí es necesaria la posición del robot. 

El problema de determinar la posición del robot se conoce como localización y consiste en la obtención de la configuración del robot a partir de un mapa o alguna representación del ambiente y un conjunto de lecturas de los sensores. La odometría se refiere al cálculo de posición únicamente mediante la integración de velocidades o aceleraciones y se utiliza cuando no se dispone de un mapa o de los sensores adecuados. 

\end{document}

%%% Local Variables:
%%% mode: latex
%%% TeX-master: t
%%% End:
