\documentclass[a4paper,12pt]{article}
\usepackage[spanish]{babel}
\spanishdecimal{.}
\selectlanguage{spanish}
\usepackage[spanish,onelanguage,ruled]{algorithm2e}
\usepackage[utf8]{inputenc}
\usepackage{graphicx}
\usepackage{caption}
\usepackage{subcaption}
\usepackage[top=2cm, bottom=2cm, left=2cm, right=2cm]{geometry}
\usepackage{hyperref}
\usepackage{verbatim}
\usepackage{amssymb}
\usepackage{mathtools}
\usepackage{listings}
\usepackage{color}
\definecolor{backcolour}{rgb}{0.95,0.95,0.92}
\newcommand\ddfrac[2]{\frac{\displaystyle #1}{\displaystyle #2}}
\lstset{backgroundcolor=\color{backcolour}, basicstyle=\footnotesize}
\lstset{xleftmargin=1cm, xrightmargin=1cm, breaklines=true}

\title{Práctica 10 \\ Seguimiento de una fuente de luz}
\author{Laboratorio de Bio-Robótica}
\date{Construcción de Robots Móviles}
\begin{document}
\renewcommand{\tablename}{Tabla}
\maketitle
\section*{Objetivos}
\begin{itemize}
\item Familiarizar al alumno con el uso del software de control de versiones \texttt{git}.
\item Aprender a utilizar el software desarrollado en el Laboratorio de Bio-Robótica para la operación de robots móviles autónomos. 
\end{itemize}

\section{Introducción}
El paradigma de la robótica basada en comportamientos afirma que un robot puede tener un comportamiento inteligente sin necesidad de una representación interna del ambiente, sino simplemente con un conjunto grande de pares entrada-salida interconectados entre sí. Dicho de otro modo, no es necesario implementar algoritmos de toma de decisiones muy complejos, sino que la inteligencia se obtiene como propiedad emergente de la interacción de un número grande de pares entrada-salida. 

A cada par entrada-salida se le conoce como comportamiento. En términos de primitivas de la robótica, un comportamiento relaciona el sensar con el actuar de una forma directa, es decir, sin que de por medio se tenga una planeación que dependa de una representación del ambiente. 

La robótica basada en comportamientos basa sus postulados


\end{document}

%%% Local Variables:
%%% mode: latex
%%% TeX-master: t
%%% End:
